\documentclass[a4paper]{article}

\usepackage[english]{babel}
\usepackage[utf8]{inputenc}
\usepackage{amsmath}
\usepackage{graphicx}
\usepackage{titling}
\setlength{\droptitle}{-6em} 
\usepackage[colorinlistoftodos]{todonotes}
\usepackage{amssymb}

\newcommand{\R}{\mathbb{R}}

\title{Factorization of second degree polynomials}

\author{Eirik Kvalheim}

\date{October 4, 2018}

\begin{document}
\maketitle

\begin{abstract}
This document contains a short explanation of a fast method for finding roots of second degree polynomials.
\end{abstract}

\section{Method for finding real roots}
\label{sec:introduction}

Let $a$, $b$, $c$, $d$, $e$ $\in$ $\R$.\\
Then the second degree polynomial equation

\begin{equation}  \label{eq:pol}
	ax^2 + bx + c = 0
\end{equation}

has the solutions

\begin{equation} \label{eq:2nd}
	\begin{array}
		{*{20}c} {x = \frac{{ - b \pm \sqrt {b^2 - 4ac} }}{{2a}}} \\ 
	\end{array}
\end{equation}

thus we can factorize (\ref{eq:pol}) as

\begin{equation} \label{eq:fac}
 (x-d)(x-e) = 0
\end{equation}

since
\begin{equation} \label{eq:Chr}
	 (x-d)(x-e)=x^2-(d+e)x+de
	%(x = \pm  e) \wedge (x = \pm  d)
\end{equation}

where $d$ and $e$ are the solutions to the right hand side of equation \ref{eq:2nd} (assuming (\ref{eq:pol}) has real roots).\\ \newline Simplifying
(\ref{eq:2nd}) by hand often takes time and is in most cases not necessary when the roots are real.
Instead we can use the following logical deductions.\\

Start by making sure the coefficient in front of $x^2$ is one:

%\begin{equation}  \label{eq:facto}
	\begin{align}
		a\Bigl( x^2 + \frac{b}{a}x + \frac{c}{a} \Bigr) = 0,
	\end{align}
%\end{equation}

so

%\begin{equation}  
	\begin{align} \label{eq:factor}
  		a = 0 \vee \Bigl( x^2 + \frac{b}{a}x + \frac{c}{a} \Bigr) = 0.
	\end{align}	
%\end{equation}

In most cases $a=1$, and we get

%\begin{equation}  \label{eq:poly}
	\begin{align}
		x^2 + bx + c = 0
	\end{align}
%\end{equation}

thus we can ask ourselves the question:\\ \newline \emph{\textbf{"What can one multiply in order to get c, and simultaneously add together to get b?"}}\\

It is important to note that the numbers searched for in this question should be put into (\ref{eq:fac}) and are \textbf{not} the roots.




\newpage

\vspace*{-30mm}

\subsection{Examples}
Consider the equation

\begin{equation}  \label{eq:ex1}
	x^2 + 6x + 8 = 0.
\end{equation}

We start by looking for two numbers to multiply in order to get $c=8$.

+4 and +2 are good candidates, and if we add them together we get $b=6$.

Thus

\begin{equation} \label{eq:ex11}
 (x+4)(x+2) = 0,
\end{equation}
and finding the roots is trivial.\\ \\
Now let

\begin{equation}  \label{eq:ex2}
	x^2 + 4x - 5  = 0.
\end{equation}

The first candidates that may come easily to mind for $c$ are $-1\cdot5$ and $1\cdot-5$.

Adding $+5$ together with $-1$ gives $b=4$.

Thus
\begin{equation} \label{eq:ex21}
 (x+5)(x-1) = 0,
\end{equation}
and finding the roots is trivial.\\ \\
Additionally we can look at

\begin{equation}  \label{eq:ex3}
	x^2 - 9x + 20 = 0.
\end{equation}

Here we can help ourself by factorizing $c$, which is our first target. 

$20=2\cdot2\cdot5$. We need to add some of the factors together to get a negative 

$b$, and we se that $-4$ and $-5$ gives $b=-9$.

Thus
\begin{equation} \label{eq:ex31}
 (x-5)(x-4) = 0,
\end{equation}
and finding the roots is trivial.\\ \\
Going further, we have

\begin{equation}  \label{eq:ex4}
	x^2 - 20x - 69 = 0.
\end{equation}

Using our technique from (\ref{eq:ex3}) we get $c=-2\cdot2\cdot5\cdot3+3\cdot3 = -3\cdot(20+3)$.

We see that $-23$ and $+3$ gives $b=-20$.

Thus
\begin{equation} \label{eq:ex41}
 (x+3)(x-23) = 0,
\end{equation}
and finding the roots is trivial.\\ \\
Finally, let

\begin{equation}  \label{eq:ex5}
	x^2 - \frac{1}{2}x - \frac{1}{2} = 0.
\end{equation}

Using our technique from (\ref{eq:ex2}) we get $c=1\cdot- \frac{1}{2}$ or $c=-1\cdot\frac{1}{2}$.

We see that $-1$ added with $\frac{1}{2}$ gives $b=-\frac{1}{2}$.

Thus
\begin{equation} \label{eq:ex51}
 (x-1)(x+\frac{1}{2}) = 0,
\end{equation}
and finding the roots is trivial.

\newpage

\subsection{Exercises}

Additional interesting problems are given below. Remember that in the cases where $a \neq 1$, the question regards \emph{what we can multiply in order to get $\frac{c}{a}$ and simultaneously add together to get $\frac{b}{a}$}.


\begin{equation}
	x^2 - 4 = 0
\end{equation}

\begin{equation}
	x^2 - 57x = 0
\end{equation}

\begin{equation}
	x^2 - 15x +26 = 0
\end{equation}

\begin{equation}
	x^2 + 14x +45 = 0
\end{equation}

\begin{equation}
	x^2 + 10x -24 = 0
\end{equation}

\begin{equation}
	x^2 - 13x +12 = 0
\end{equation}

\begin{equation}
	x^2 + 3x -70 = 0
\end{equation}

\begin{equation}
	x^2 -12x +35 = 0
\end{equation}

\begin{equation}
	2x^2 -x -21 = 0
\end{equation}

\begin{equation}  \label{eq:ex5}
	3x^2 + 2x + \frac{1}{3} = 0
\end{equation}

\begin{equation}
	2x^2 -8x -24 = 0
\end{equation}

\begin{equation}
	3x^2 -11x -4 = 0
\end{equation}

\begin{equation}
	3x^2 -14x +5 = 0
\end{equation}

\begin{equation}
	8x^2 +14x -15 = 0
\end{equation}

\begin{equation}
	5x^2 -34x +24 = 0
\end{equation}

\begin{equation}
	8x^2 -47x -63 = 0
\end{equation}

\begin{equation}
	6x^2 +11x -35 = 0
\end{equation}

\begin{equation}
	11x^2 +18x -7 = 0
\end{equation}


%\begin{thebibliography}{9}
%\bibitem{nano3}
% Eirik Kvalheim,
%  \emph{Robotikk Anvendelser}.
% Universitetet i Oslo \& Kuben Yrkesarena
%
%\end{thebibliography}

\end{document}